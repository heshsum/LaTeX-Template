\documentclass[a4paper, 11pt]{article}

% Paket für verlinkte Querverweise
\usepackage[hidelinks]{hyperref}

% Pakete für Zeichensatz und Sprache
\usepackage[T1]{fontenc}
\usepackage[utf8]{inputenc}
\usepackage[ngerman]{babel}

% Nutze einen Font ohne Serifen
\usepackage{helvet}
\renewcommand{\familydefault}{\sfdefault}

% Paket für Bibliografie
\usepackage[style=authoryear, backend=biber]{biblatex}
\addbibresource{bibliography.bib}
\usepackage{csquotes}

% Einstellungen für Seitenlayout
\usepackage[left=2.5cm, right=2.5cm, top=2.5cm, bottom=2.5cm]{geometry}

% Paket für Absatzformatierung
\usepackage{parskip}

% Paket für Zeilenabstand
\usepackage{setspace}
\onehalfspacing

% Paket damit Tabellen dort landen wo sie sollen
\usepackage{float}

% Paket für Kopf- und Fußzeilen
\usepackage{fancyhdr}
\pagestyle{fancy}
\fancyhf{}
\rfoot{\thepage}
\renewcommand{\headrulewidth}{0pt}

% Paket für Titelseite
\usepackage{titling}

% Paket für Tabellenzellen mit Zeilenumbruch
\usepackage{makecell}

\usepackage{graphicx}

% Paket für Glossar
\usepackage[acronym]{glossaries}
\makenoidxglossaries

% Glossareinträge
\newglossaryentry{latex}
{
    name=latex,
    description={Is a markup language specially suited 
    for scientific documents}
}
\newacronym{abk}{Abk.}{Abkürzung. Wird in der Liste der Akronyme aufgeführt.}

% Titelinformationen
\selectlanguage{ngerman}
\title{Titel der Arbeit}
\author{Heiner Hansen}
\date{\today}

\begin{document}

% Titelseite
\begin{titlepage}
    \begin{center}
        \vspace*{1cm}
        \Huge
        \textbf{\thetitle}
        
        \vspace{0.5cm}
        \LARGE
        Dies ist der Untertitel
        
        \vspace{1.5cm}
        
        \Large
        verfasst von \\
        \theauthor \\
        Matrikelnummer: 123456
        
        \vfill
        
        betreut von \\
        Prof. Dr. Pi Rogge \\
        Berliner Hochschule für Technik \\
        Von Leitner Institut für verteiltes Echtzeit-Java
        
        \vspace{1.5cm}
        
        \Large
        \thedate
        
    \end{center}
\end{titlepage}

% Inhaltsverzeichnis
% Keine Seitenzahlen beim Inhaltsverzeichnis
\pagenumbering{gobble}
\tableofcontents
\newpage

% Abbildungsverzeichnis und Tabellenverzeichnis
\pagenumbering{Roman}
\listoffigures
% Tabellenverzeichnis
\listoftables
\newpage

% Glossar
\printnoidxglossary[type=\acronymtype,title=Abkürzungsverzeichnis]
\newpage

%Sperrvermerk ohne Seitenzahlen
\section*{Sperrvermerk}  \label{section:sperrvermerk}
% Kapitel ohne Kapitelnummer

Diese Arbeit enthält vertrauliche Daten der Schneidereit GmbH. Eine Weitergabe der Arbeit im Ganzen oder in Teilen sowie das Anfertigen von Kopien (auch digital) sind untersagt.

Ausnahmen bedürfen der schriftlichen Genehmigung. 

Schneidereit GmbH\\
Eine Straße 100\\
10199 Berlin

Kontakt:\\
mail@schneidereit.internal\\
+49 30 123456-9

Berlin, \thedate, \author{}

% Seitenzahlen ab hier, Start bei 1
\pagenumbering{arabic}
\setcounter{page}{1}

% Beginn des Hauptteils
\section{Einleitung} \label{section:einleitung}
Dies ist die Einleitung.

"Dies ist ein Zitat" \parencite[Vgl.][32f]{knuthwebsite}.

Eine andere Referenz \parencite{einstein}

Dies ist ein Akronym: \acrlong[]{abk}

Dies ist ein \\
Umbruch.

Dies ist eine Referenz: Kapitel \ref{section:fazit}

Dies ist eine Tabelle:

\begin{table}[H]
    \centering
    \begin{tabular}{|c|c|c|} \hline
        \textbf{Überschrift 1}  & \textbf{Überschrift 2}    & \textbf{Überschrift 3} \\ 
        \hline
        Wert 1                  & Wert 2                    & Wert 3 \\
        \hline
    \end{tabular}
    \caption{Tabelle}
\end{table}
\section{Fazit} \label{section:fazit}
Dies ist das Fazit. Darin werden die wichtigsten Erkenntnisse zusammengefasst.

\medskip
\newpage

% Literaturverzeichnis
\newpage
\printbibliography[]
\newpage

\section*{Erklärung der Selbstständigkeit}

Hiermit versichere ich, dass ich die vorliegende Arbeit selbstständig, ohne unzulässige Hilfe Dritter und ohne Benutzung anderer als der angegebenen Hilfsmittel angefertigt habe.

Die aus fremden Quellen direkt oder indirekt übernommenen Gedanken sind als solche kenntlich gemacht. 

Berlin, den \thedate

% Arbeitsbescheinigung
\newpage
\section*{Anhang}
% Anhang als Inline-Datei
% \includegraphics[scale=0.8]{Datei.pdf}

\end{document}
